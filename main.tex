% Project created on 2020-07-24
% Author: Alex Rutar (arutar@uwaterloo.ca)
\documentclass[12pt]{article}
\usepackage[bottom=3cm, right=2.5cm, top=2.5cm, left=2.5cm]{geometry}

% style packages
\usepackage{.texproject/styles/local-palatino-header}

% macro packages
\usepackage{.texproject/macros/local-general}
\usepackage{.texproject/macros/local-typesetting}
\usepackage{.texproject/macros/local-refs_ams}
\usepackage{.texproject/macros/local-theorem_ams}
\usepackage{project-macros}

\begin{document}
\par{\centering
    {\Huge \textbf{Curriculum Vitae}\\\vspace{0.5cm}\huge \textit{Alex Rutar}
}\bigskip\par}
\noindent

\textit{Last updated:} \textit{\today}


\begin{twocolsec}{Personal Information}
    \textbf{Institution}   & University of St Andrews \\
    \textbf{Email}     & \href{mailto:alex@rutar.org}{\texttt{alex@rutar.org}}\\
    \textbf{Website}     & \href{https://rutar.org}{\texttt{https://rutar.org}}\\
    \textbf{Citizenship} & Canadian\\
    \textbf{Languages}   & English (native), French (reading)
\end{twocolsec}


\begin{twocolsec}{Education}
    2020-     & \textbf{PhD in Mathematics}, {\bfseries\textit{University of St Andrews}}, \textit{St Andrews, UK}\\
                & Advisors: Jonathan Fraser and Kenneth Falconer\\&\\
    2016-2020 & \textbf{Bachelor of Mathematics}, {\bfseries\textit{University of Waterloo}}, \textit{Waterloo, ON}\\
                & Major: Pure Mathematics, Minor: Combinatorics and Optimization\\
                & GPA: 95.7/100\\&\\
    Fall 2018 & \textbf{Exchange}, {\bfseries\textit{Budapest Semesters in Mathematics}}, \textit{Budapest}\\
                & Magna Cum Laude\\
                & GPA: 4.0/4.0\\&\\
    2012-2016 & \textbf{Secondary School}, {\bfseries\textit{Tempo School}}, \textit{Edmonton, AB}\\
                & Advanced Placement National Scholar\\
                & GPA: 99/100
\end{twocolsec}

\begin{threecolsec}{Funding}
    2021 & £15,609 & \textbf{EPSRC Doctoral Funding}\\
    2020 & £15,285 & \textbf{EPSRC Doctoral Funding}\\
    2019 & \$4,500 & \textbf{NSERC Undergraduate Research Award}\\
    2018 & \$4,500 & \textbf{NSERC Undergraduate Research Award}
\end{threecolsec}

\begin{threecolsec}{Scholarships and Awards}
    2020 & £73,000 & \textbf{Hansel Scholarship}, \textit{University of St Andrews}\\
    2020 & \$1,000 & \textbf{Pure Math Undergraduate Research Prize}, \textit{University of Waterloo}\\
    2016 & \$20,000 & \textbf{W. T. Tutte National Scholarship}, \textit{University of Waterloo}\\
    2016 & \$5,000 &\textbf{President’s Scholarship}, \textit{University of Waterloo}\\
    2016 & \$2,500 &\textbf{Rutherford Scholarship}, \textit{Government of Alberta}\\
    2016 & \$0 &\textbf{Governor General Bronze}, \textit{Tempo School}
\end{threecolsec}


\begin{ensec}{Publications}
\item A. Banaji, A. Rutar. Attainable forms of intermediate dimensions. \textit{\href{https://arxiv.org/abs/2111.14678}{arXiv:2111.14678} (submitted)}.
\item A. Rutar. A Multifractal Decomposition for Self-similar Measures with Exact Overlaps. \textit{\href{https://arxiv.org/abs/2104.06997}{arXiv:2104.06997} (submitted)}.
\item K. E. Hare, A. Rutar. Local Dimensions of Self-similar Measures Satisfying the Finite Neighbour Condition. \textit{\href{https://arxiv.org/abs/2101.07400}{arXiv:2101.07400} (submitted)}.
\item A. Rutar. Geometric and Combinatorial Properties of Self-similar Multifractal Measures. \textit{\href{https://arxiv.org/abs/2008.00197}{arXiv:2008.00197} (submitted)}.
\item K. E. Hare, K. G. Hare, A. Rutar. When the Weak Separation Condition implies the Generalized Finite Type Condition. \textit{Proc. Amer. Math. Soc.} 149 (2021), 1555-1568.
\end{ensec}

\begin{twocolsec}{Conferences and Presentations}
    Feb. 2022 & \textbf{St Andrews Analysis Seminar}: \textit{Attainable forms of intermediate dimensions}\\
    Apr. 2021 & \textbf{Junior Ergodic Theory Seminar}: \textit{Self-similar measures with non-concave spectra and multifractal analysis}\\
    Jan. 2021 & \textbf{St Andrews Online Burn Meet}: \textit{Analysis Group Intro Talk}\\
    Oct. 2020 & \textbf{St Andrews Analysis Seminar}: \textit{Multifractal Analysis for Self-Similar Measures with Exact Overlaps}\\
    Feb. 2020 & \textbf{Waterloo Analysis Seminar}: \textit{Geometric and Combinatorial Separation Conditions for Iterated Function Systems}\\
    Jul. 2019 & \textbf{CUMC 2019}: \textit{An Algebraic Proof of Quadratic Reciprocity}\\
    Jul. 2018 & \textbf{CUMC 2018}: \textit{Pisot–Vijayaraghavan numbers}
\end{twocolsec}

\begin{twocolsec}{Other Skills}
    \LaTeX & typesetting and package development\\
    Python & software development, numerical computation, symbolic computation, graphical tools\\
    Mathematica & functional programming, algorithm implementation for research papers
\end{twocolsec}

\end{document}

